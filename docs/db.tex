
\documentclass{article} % For LaTeX2e
\usepackage{iclr2021_conference,times}

% Optional math commands from https://github.com/goodfeli/dlbook_notation.
\input{math_commands.tex}

\usepackage{hyperref}
\usepackage{url}
\usepackage{amsthm}

\title{$\partial\mathbb{B}$ nets: learning boolean functions with\\backpropagation}

% Authors must not appear in the submitted version. They should be hidden
% as long as the \iclrfinalcopy macro remains commented out below.
% Non-anonymous submissions will be rejected without review.

\author{Ian Wright \thanks{GitHub, Z80coder@github.com}
}

% The \author macro works with any number of authors. There are two commands
% used to separate the names and addresses of multiple authors: \And and \AND.
%
% Using \And between authors leaves it to \LaTeX{} to determine where to break
% the lines. Using \AND forces a linebreak at that point. So, if \LaTeX{}
% puts 3 of 4 authors names on the first line, and the last on the second
% line, try using \AND instead of \And before the third author name.

\newcommand{\fix}{\marginpar{FIX}}
\newcommand{\new}{\marginpar{NEW}}

\newtheorem{theorem}{Theorem}
\newtheorem*{definition}{Definition}
\newtheorem{prop}{Proposition}


\iclrfinalcopy % Uncomment for camera-ready version, but NOT for submission.
\begin{document}

\maketitle

\begin{abstract}
	$\partial\mathbb{B}$ nets are real-valued, differentiable neural networks 
	trained by backpropagation that yield boolean functions with identical accuracy.
	Existing approaches to neural network binarization lose accuracy. In contrast, 
	$\partial\mathbb{B}$ nets `harden' to boolean functions with identical semantics.  Experiments demonstrate that $\partial\mathbb{B}$ nets achieve competitive accuracy on standard machine learning problems yet are significantly more compact and interpretable.
\end{abstract}

\section{Introduction}

\section{Related work}

\section{$\partial\mathbb{B}$ nets}

\begin{definition}[Soft-bits and hard-bits]
A {\em soft-bit} is a real value in the range $[0,1]$ and a {\em hard-bit} is a boolean value in the set $\{0,1\}$. A soft-bit, $x$, is {\em high} if $x>1/2$, otherwise it is {\em low}.
\end{definition}

\begin{definition}[Hardening]
The {\em hardening} function, $h(x_{1}, \dots, x_{n}) = [f(x_{1}), \dots, f(x_{n})]$, converts soft-bits to hard-bits, where
\begin{equation*}
f(x) =
\begin{cases}
1 & \text{if } x > 1/2 \\
0 & \text{otherwise.}
\end{cases}
\end{equation*}
\end{definition}

\begin{definition}[Hard-equivalence]
	A function, $f: [0,1]^n \rightarrow [0,1]^m$, is {\em hard-equivalent} to a boolean function, $g: \{1,0\}^n \rightarrow \{1,0\}^m$,	if
	$h(f(h({\bf x}))) = g(h({\bf x}))$ for all ${\bf x} \in [0,1]^{n}$.
\end{definition}

\subsection{Differentiable boolean functions}

Weights are trainable soft-bits. A high weight implies the corresponding operation is masked out and therefore inactive.

\begin{definition}
$\partial\text{NOT}: [0, 1]^{2} \rightarrow [0,1]$ is the function,
	\begin{equation*}
	\partial\text{NOT}(w, x) = 1 - w + x (2w - 1)\text{,}
	\end{equation*}
where $w$ is a weight and $x$ is a soft-bit value.
\end{definition}

\begin{prop}\label{prop:not}
	$\partial${NOT} is hard-equivalent to the boolean function
$\neg (x \oplus w)$.
\end{prop}

If the weight is high then $\partial${NOT} is hard-equivalent to the boolean identity function; otherwise it is hard-equivalent to $\neg$. In consequence, we can learn to logically not, or simply pass through, the input value $x$.

\begin{definition}
$\partial\text{AND}: [0,1]^{2} \rightarrow [0,1]$ is the function,
	\begin{equation*}
	\partial\text{AND}(x, y) = 
	\begin{cases}
	1/2 + 1/2(x + y)(m - 1/2) & \text{if } 2m > 1 \\
	m + 1/2(x + y)(1/2 - m) & \text{otherwise,}
	\end{cases}
	\end{equation*}
	where $m=\min(x,y)$, and $x$ and $y$ are soft-bit values.
\end{definition}

\begin{prop}\label{prop:and}
	$\partial${AND} is hard-equivalent to the boolean function $x \wedge y$.
\end{prop}

\begin{definition}
$\partial\text{OR}: [0,1]^{2} \rightarrow [0,1]$ is the function,
	\begin{equation*}
	\partial\text{OR}(x, y) = 
	\begin{cases}
	1/2 + 1/2(x + y)(m - 1/2) & \text{if } 2m > 1 \\
	m + 1/2(x + y)(1/2 - m) & \text{otherwise,}
	\end{cases}
	\end{equation*}
	where $m=\max(x,y)$, and $x$ and $y$ are soft-bit values.
\end{definition}

\begin{prop}\label{prop:or}
	$\partial${OR} is hard-equivalent to the boolean function $x \vee y$.
\end{prop}

\begin{definition}
$\partial\text{IMPLIES}: [0,1]^{2} \rightarrow [0,1]$ is the function,
	\begin{equation*}
	\partial\text{IMPLIES}(w, x) = \partial\text{OR}(x, 1-w)\text{,}
	\end{equation*}
where $w$ is a weight and $x$ is a soft-bit value.
\end{definition}

\begin{prop}\label{prop:implies}
	$\partial${IMPLIES} is hard-equivalent to the boolean function $w \Rightarrow x$.
\end{prop}

Define $\operatorname{majority-index}: \mathbb{Z}_{>0} \rightarrow \mathbb{Z}_{> 0}$ as $\operatorname{majority-index}(n) = 1 + \lfloor \frac{n-1}{2}\rfloor$.

Define $\operatorname{select}: [0,1]^n \times {1, 2, \ldots, n} \rightarrow [0,1]$ as  $\operatorname{select}({\bf x}, i) = x_{i}$.

Define $\operatorname{majority-bit}: [0,1]^n \rightarrow [0,1]$ as  $$\operatorname{majority-bit}({\bf x}) = \operatorname{select}( \operatorname{sort}({\bf x}), \operatorname{majority-index}(\lvert{\bf x}\rvert))\text{,}$$ where $\operatorname{sort}$ sorts the elements of a vector in ascending order.

Define $\operatorname{majority-delta}: [0,1]^n \rightarrow [0,1]$ as
$$\operatorname{majority-delta}({\bf x}) = \bar{\bf x} \times \left| \operatorname{majority-bit}({\bf x}) - 1/2\right|\text{.}$$

\begin{definition}
$\partial\text{MAJORITY}: [0,1]^{n} \rightarrow [0,1]$ is the function,
	\begin{equation*}
	\partial\text{MAJORITY}({\bf x}) = 
	\begin{cases}
	1/2 + \delta & \text{if } m > 1/2 \\
	m + \delta & \text{otherwise,}
	\end{cases}
	\end{equation*}
	where ${\bf x}$ is a vector of soft-bits, $m = \operatorname{majority-bit}({\bf x})$ and $\delta = \operatorname{majority-delta}({\bf x})$.
\end{definition}

\begin{prop}\label{prop:majority}
	$\partial${MAJORITY} is hard-equivalent to the boolean majority function,
\begin{equation*}
\text{MAJ}(x_{1}, \dots, x_{n}) = 
	\begin{cases}
	1 & \text{if } \sum_{i=1}^{n} x_{i} \geq n/2 \\
	0 & \text{otherwise.}
\end{cases}
\end{equation*}
\end{prop}


\subsection{Logical layers}



\section{Experiments}

\section{Conclusion}

\begin{figure}[h]
\begin{center}
%\framebox[4.0in]{$\;$}
\fbox{\rule[-.5cm]{0cm}{4cm} \rule[-.5cm]{4cm}{0cm}}
\end{center}
\caption{Sample figure caption.}
\end{figure}


\subsubsection*{Acknowledgments}
Use unnumbered third level headings for the acknowledgments. All
acknowledgments, including those to funding agencies, go at the end of the paper.


\bibliography{iclr2021_conference}
\bibliographystyle{iclr2021_conference}

\appendix

\section*{Appendix}

\section{Proofs}

\subsection{Proof of proposition~\ref{prop:not}}

\begin{proof}
	Table \ref{not-table} is the truth table of the boolean function $\neg (x \oplus w)$.
	\begin{table}
		\begin{center}
			\begin{tabular}{lll}
				\multicolumn{1}{c}{$h(w)$}  &\multicolumn{1}{c}{$h(x)$} &\multicolumn{1}{c}{$\partial\text{NOT}(h(w), h(x))$}
				\\ \hline \\
				0 & 0 & 1\\
				1 & 0 & 0\\
				0 & 1 & 0\\
				1 & 1 & 1\\
			\end{tabular}
		\end{center}
		\caption{$\partial${NOT} is hard-equivalent to $\neg (x \oplus w)$.}\label{not-table}
		
	\end{table}
\end{proof}

\subsection{Proof of proposition~\ref{prop:and}}

\begin{proof}
	Table \ref{and-table} is the truth table of the boolean function $x \wedge y$.
	\begin{table}
		\begin{center}
			\begin{tabular}{llll}
				\multicolumn{1}{c}{$h(x)$}  &\multicolumn{1}{c}{$h(y)$} &\multicolumn{1}{c}{$\partial\text{AND}(h(x), h(y))$} &\multicolumn{1}{c}{$h(\partial\text{AND}(h(x), h(y)))$}
				\\ \hline \\
				1 & 1 & 1 & 1\\
				1 & 0 & 1/4 & 0\\
				0 & 1 & 1/4 & 0\\
				0 & 0 & 0 & 0\\
			\end{tabular}
		\end{center}
		\caption{$\partial${AND} is hard-equivalent to $x \wedge y$.}\label{and-table}
		
	\end{table}
\end{proof}

\subsection{Proof of proposition~\ref{prop:or}}

\begin{proof}
	Table \ref{or-table} is the truth table of the boolean function $x \vee y$.
	\begin{table}
		\begin{center}
			\begin{tabular}{llll}
				\multicolumn{1}{c}{$h(x)$}  &\multicolumn{1}{c}{$h(y)$} &\multicolumn{1}{c}{$\partial\text{OR}(h(x), h(y))$} &\multicolumn{1}{c}{$h(\partial\text{OR}(h(x), h(y)))$}
				\\ \hline \\
				1 & 1 & 1 & 1\\
				1 & 0 & 3/4 & 1\\
				0 & 1 & 3/4 & 1\\
				0 & 0 & 0 & 0\\
			\end{tabular}
		\end{center}
		\caption{$\partial${OR} is hard-equivalent to $x \vee y$.}\label{or-table}
		
	\end{table}
\end{proof}

\subsection{Proof of proposition~\ref{prop:implies}}

\begin{proof}
	Table \ref{implies-table} is the truth table of the boolean function $x \Rightarrow y$.
	\begin{table}
		\begin{center}
			\begin{tabular}{llll}
				\multicolumn{1}{c}{$h(w)$}  &\multicolumn{1}{c}{$h(x)$} &\multicolumn{1}{c}{$\partial\text{IMPLIES}(h(w), h(x))$} &\multicolumn{1}{c}{$h(\partial\text{IMPLIES}(h(w), h(x)))$}
				\\ \hline \\
				1 & 1 & 3/4 & 1\\
				1 & 0 & 0 & 0\\
				0 & 1 & 1 & 1\\
				0 & 0 & 3/4 & 1\\
			\end{tabular}
		\end{center}
		\caption{$\partial${IMPLIES} is hard-equivalent to $w \Rightarrow x$.}\label{implies-table}
		
	\end{table}
\end{proof}

\subsection{Proof of proposition~\ref{prop:majority}}

\begin{proof}
todo
\end{proof}

\end{document}
